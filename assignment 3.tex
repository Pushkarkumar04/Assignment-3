\documentclass{article}
\usepackage[utf8]{inputenc}
\usepackage{graphicx}

\title{ASSIGNMENT 3\\ON\\ Future Of Healthcare}
\author{BY\\PUSHKAR KUMAR\\ROLL NO.: 21111040\\FIRST SEMESTER\\BRANCH:BIOMEDICAL ENGINEERING\\SECTION:A\\NATIONAL INSTITUTE OF TECHNOLOGY, RAIPUR\\ASSIGNMENT SUBMITTED TO\\
DEPARTMENT OF BIOMEDICAL ENGINEERING}
\date{}

\begin{document}

\maketitle
\begin{figure}[h]
    \centering
    \includegraphics[height=9cm,width=9cm]{download.jpg}
\end{figure}


\section{Future Of Healthcare}
The future of health will likely be driven by digital transformation enabled by radically interoperable data and open, secure platforms. Health is likely to revolve around sustaining well-being rather than responding to illness.TWENTY years from now, cancer and diabetes could join polio as defeated diseases. We expect prevention and early diagnoses will be central to the future of health. The onset of disease, in some cases, could be delayed or eliminated altogether. Sophisticated tests and tools could mean most diagnoses (and care) take place at home.Interventions and treatments are likely to be more precise, less complex, less invasive, and cheaper. Health will be defined holistically as an overall state of well-being encompassing mental, social, emotional, physical, and spiritual health. Not only will consumers have access to detailed information about their own health, they will own their health data and play a central role in making decisions about their health and well-being.The future of healthcare lies in working hand-in-hand with technology and healthcare workers have to embrace emerging healthcare technologies in order to stay relevant in the coming years.
\subsection{ Artificial intelligence}
I believe that artificial intelligence has the potential to redesign healthcare completely. AI algorithms are able to mine medical records, design treatment plans or create drugs way faster than any current actor on the healthcare palette including any medical professional. 
\subsection{Virtual reality}
Virtual reality (VR) is changing the lives of patients and physicians alike. In the future, you might watch operations as if you wielded the scalpel or you could travel to Iceland or home while you are lying on a hospital bed. VR is being used to train future surgeons and for actual surgeons to practice operations. Such software programmes are developed and provided by companies like Osso VR and ImmersiveTouch and are in active use with promising results. 
\subsection{Augmented reality}
Augmented reality differs from VR in two respects: users do not lose touch with reality and it puts information into eyesight as fast as possible. These distinctive features enable AR to become a driving force in the future of medicine; both on the healthcare providers’ and the receivers’ side.In case of medical professionals, it might help medical students prepare better for real-life operations, as well as enables surgeons to enhance their capabilities.
\subsection{Healthcare trackers, wearables and sensors}
As the future of medicine and healthcare is closely connected to the empowerment of patients as well as individuals taking care of their own health through technologies, I cannot leave out health trackers, wearables and sensors from my selection. They are great devices to get to know more about ourselves and retake control over our own lives.With the ability to monitor one’s health at home and share the results remotely with their physician, these devices empower people to take control of their health and make more informed decisions.
\subsection{Medical tricorder}
When it comes to gadgets and instant solutions, there is the great dream of every healthcare professional: to have one almighty and omnipotent device, with which you can diagnose and analyze every disease. It even materialized – although only on screen – as the medical tricorder in Star Trek. 
\subsection{Revolutionizing drug development}
Currently, the process of developing new drugs is too long and too expensive. However, there are ways to improve drug development with methods ranging from artificial intelligence to in silico trials. Such new technologies and approaches already are and will be dominating the pharmaceutical landscape in the years to come.
\subsection{Nanotechnology}
We are living at the dawn of the nanomedicine age. I believe that nanoparticles and nanodevices will soon operate as precise drug delivery systems, cancer treatment tools or tiny surgeons.As the technology evolves, we will see more practical examples of nanotechnology in medicine. Future PillCams could even take biopsy samples for further analysis while remote-controlled capsules could make the prospect of nano-surgeons a reality.
\subsection{Robotics}
One of the most exciting and fastest growing fields of healthcare is robotics; developments range from robot companions through surgical robots until pharmabotics, disinfectant robots or exoskeletons. There are loads of other applications for these sci-fi suits from aiding nurses through lift elderly patients to helping patients with spinal cord injury.
 \subsection{3D-printing}
 3D-printing can bring wonders in all aspects of healthcare. We can now print biotissues, artificial limbs, pills, blood vessels and the list goes on and will likely keep on doing so.




\end{document}